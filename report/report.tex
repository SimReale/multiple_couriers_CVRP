\documentclass{article}
\usepackage{array}
\usepackage{graphicx}
\usepackage{amsmath, amsthm, thmtools, mathdots, mathtools, amssymb}
\usepackage[pdfusetitle]{hyperref}
\usepackage[all]{hypcap} % Links hyperref to object top and not caption
\usepackage[nameinlink]{cleveref}   
\usepackage{makecell, multirow, booktabs}
\usepackage{caption, subcaption}
\usepackage{float}
\usepackage[bottom]{footmisc}
\usepackage[inline]{enumitem}
\usepackage{appendix}
\usepackage{biblatex}
\addbibresource{references.bib}
\hypersetup{ colorlinks, citecolor=black, filecolor=black, linkcolor=black, urlcolor=black, linktoc=all }
\newcommand*\rot{\rotatebox{90}}

\let\endtitlepage\relax

\newtheorem{theorem}{Theorem}
\newtheorem{definition}{Definition}
\newtheorem{lem}{Claim}


\begin{document}
    \begin{titlepage}
        \begin{center}
            {\LARGE Multiple Couriers Problem}
            \vspace*{1em}
            
            Valerio Costa, Luca Domeniconi, Claudia Maiolino, Tian Cheng Xia

            \centerline{\{valerio.costa, luca.domeniconi5, claudia.maiolino, tiancheng.xia\}@studio.unibo.it}
        \end{center}
    \end{titlepage}

    \thispagestyle{plain}

    \section{Introduction} \label{sec:intro}
    The problem of this project is known in the literature as the capacitated vehicle routing problem and can be easily proven to be NP-hard. We tackle this problem by virtually splitting it into two sub-problems: first we look for an assignment of the items to the couriers and then search for the routes of each of them (i.e., by solving multiple travelling salesman problems). To solve the latter, we follow the approach presented in \cite{vrp} where the route of a courier is modelled through the variables $P_d \in [1, n+1]$ with $d \in [1, n+1]$ defined as follows:
    \begin{equation}
        \label{eq:path_def}
        P_{d_1} = \begin{cases}
            d_2 & \text{iff the location $d_2 \neq d_1$ is visited immediately after $d_1$}\\
            d_1 & \text{iff $d_1$ is not part of the route of the courier}
        \end{cases}
    \end{equation} 
    With proper assignment and subtour elimination constraints, $P_d$ allows to define a Hamiltonian cycle that passes through the items that the courier delivers and the solution can be extracted by traversing the cycle starting from the depot $n+1$.

    The lower-bound of the objective function is common to all models and is defined as the maximum path cost that involves a single package:
    \begin{equation}
        \max_{p \in [1, n]} \left\{ D_{n+1, p} + D_{p, n+1} \right\}
    \end{equation}

    As upper-bound, also common to all models, we observed that it does not provide any improvement to the results. Nevertheless, we defined it as:
    \begin{equation}
        \sum_{d_1 \in [1, n+1]} \max_{d_2 \in [1, n+1]} D_{d_1, d_2}
    \end{equation}

    Symmetry breaking constraints are also common to all models. By considering couriers with the same capacity, the following constraints can be used to avoid symmetries:
    \begin{itemize}
        \item By imposing an ordering on the amount of assigned packages:
            \begin{equation}
                \label{eq:cp_symm_amount}
                \forall c_1, c_2 \in [1, m]: (c_1 < c_2 \land l_{c_1} = l_{c_2}) \Rightarrow Q_{c_1} \leq Q_{c_2}
            \end{equation}
            where $Q_c$ is the amount of packages delivered by the courier $c$.
        \item By imposing an ordering on the indexes of the assigned packages:
            \begin{equation}
                \label{eq:cp_symm_packs}
                \forall c_1, c_2 \in [1, m]: (c_1 < c_2 \land l_{c_1} = l_{c_2}) \Rightarrow A_{c_1} <_\texttt{lex} A_{c_2}
            \end{equation}
            where $A_c$ is an ordered vector containing the packages delivered by the courier $c$.
    \end{itemize}

    As the triangle inequality holds, we also identified an implied constraint that consists of imposing that each courier delivers at least a package (a short proof is provided in \Cref{sec:impl_proof}):
    \begin{equation}
        \label{eq:impl_constr}
        \forall c \in [1, m]: Q_c \geq 1  
    \end{equation}
    where $Q_c$ is the amount of packages delivered by the courier $c$.
    This obviously is applicable only if each courier is able to carry at least an item.

    All experiments were done using the same random seed and were run as workflows on GitHub Actions which provides two cores at 2.45 GHz and 7 GB of memory. To guarantee a safe margin for the Docker container to run, the actual usable memory was capped to 5 GB.

    The work has been completed in approximately one month and has been roughly split in the following way: Xia did the CP part, Costa worked on SAT, Domeniconi did the SMT models, and Maiolino completed the MIP part. The main difficulties we encountered are the following: 
    \begin{enumerate*}[label=(\roman*)]
        \item lack of proper documentation for many tools we used,
        \item difficulties to find a more efficient way to solve bigger instances,
        \item the need of time to run the experiments on all instances.
    \end{enumerate*}


    \section{CP}
The CP models are set in an \textit{incremental way}, starting from a base structure with the foundamental constraints of the problem and then adding implied and symmetry breaking ones. After fixing the models, a final exploration on the search strategy is provided to optimize the solving procedure.



\subsection{Decision variables}
The CP models are based on the following decision variables:
\begin{itemize}
    \item $x_{c, i} \in NODES$, with $c \in COURIERS$ and $i \in NODES$, which defines the construction of the paths for each courier, meaning that $x_{c, i} = j$ if the courier $c$ goes from the node $i$ to $j$, where $n+1$ is the \textit{depot}.
    
    \item $bins_i \in COURIERS$, with $i \in ITEMS$, where setting $bins_i = c$ means that the item $i$ is assigned to the courier $c$.
    
    \item $load_c \in [min\_load\ ..\ max\_load]$, with $c \in COURIERS$, that is an \textit{auxiliary variable} used to assign the load carried by each courier in the solution, where $min\_load$ and $max\_load$ are respectively the \textit{lower bound} and the \textit{upper bound} of the load values:
    \begin{quote}
        \centering
        $min\_load = min(s)$, $max\_load = max(l)$.
    \end{quote}
\end{itemize}



\subsection{Objective function}
The \textit{objective function} is defined through the distance matrix $D$ given as input and the paths $x$ in the following way:
\begin{equation}
    \max_{c \in COURIERS} \left( \sum_{i \in NODES:\ x_{c, i} \neq i} D_{i, x_{c, i}} \right)
\end{equation}



\subsection{Constraints}
The main problem structure is defined through the following constraints.
\begin{itemize}
    \item \texttt{bin\_packing\_capa}

        This global constraint taken from the MiniZinc library ensures that the actual load carried by each courier, given by the sum of its item sizes, is lower than its overall capacity. Mathematically speaking, it is defined by:
        \begin{equation}
        \forall c \in COURIERS: \left( \sum_{i \in ITEMS:\ bins_i = c} s_i \right) \leq l_c.
        \end{equation}
        The global version has been preferred to the explicit one, due to a more optimized implementation provided by MiniZinc for performances.
    
    \item \textbf{Couriers assignment and Channelling constraint}

        It defines how the items must be assigned to the couriers and links the $bins$ and $x$ matrices in the following way:
        \begin{equation}
            \forall c \in COURIERS,
            \forall i \in ITEMS: 
            \begin{cases}
                x_{c, i} = i    & \text{if $bins_i \neq c$} \\
                x_{c, i} \neq i & \text{if $bins_i = c$} 
            \end{cases}
        \end{equation}
    
    \item \texttt{subcircuit}

        Again, it is a MiniZinc global constraint, that models the fact that $x$ is the \textit{successors matrix of each node}, guaranteeing that each courier forms an Hamiltonian cycle through its nodes assignments. Therefore, this contraint is applied on each row of the $x$ matrix.

    \item \textbf{Couriers departure}

        It states explicitly that each courier must take at least one pack and not stay in the depot, due to performance optimization:
        \begin{equation}
            \forall c \in COURIERS:\ x_{c, n+1} \neq n+1 
        \end{equation}

\end{itemize}

\subsubsection{Implied constraints}
The only implied constraint has been added to the model to improve the performance through the introduction of the variable $load$ and is \texttt{bin\_packing\_load}. It assigns the load of each courier as the sum of the items' sizes $s$, depending on the assignments in $bins$ and is defined mathematically as:
\begin{equation}
    \forall c \in COURIERS: \left( \sum_{i \in ITEMS:\ bins_i = c} s_i \right) = load_c.
\end{equation}

\subsubsection{Symmetry breaking constraints}
Experimental trials showed that too much symmetry breaking constraints cause the performance to worsen, so only one symmetry breaking about the load is left in the related model. In particular, it is the \texttt{lex\_lesseq}, which imposes the lexicographic ordering on the couriers assignments of two items if the items have the same size and the couriers have the same load carried:
\begin{quote}
    \centering
    $\forall i \in ITEMS, \forall j \in ITEMS:$
\end{quote}
\begin{equation}
    (s_i = s_j \land load[bins_i] = load[bins_j]) \implies lex\_lesseq([bins_i,\ bins_j])
\end{equation}



\subsection{Validation}

\subsubsection{Experimental design}
The models are evaluated using as solvers both \textit{Gecode} and \textit{Chuffed}, with ad-hoc search strategies, in order to fully exploit the potential of each solver. After making different experiments on the variable assignation, it came up that $x$ is the variable matrix that most influences the exploration of the search tree. However, different choices have been taken for Gecode and Chuffed, depending on the their availability of \textit{variable selection} and \textit{domain selection} strategies.
\begin{itemize}
    \item \textbf{Gecode:}

        The exploration of the search tree is managed by the \texttt{int\_default\_search}, which is the type of integer search that is properly suited on the Gecode solver. It turned out that the best annotation choices for this strategy are \texttt{afc\_size\_max} for variables, together with a random choice for the value assignment (\texttt{indomain\_random}). 

        In addition, a restart strategy has been added, based on the \textit{Luby sequence}, in combination with the \textit{Local Neighborhood Search (LNS)}. From previous experiments, we noticed that low values of the retain percentage did not influence much the exploration. Therefore, it has been set to the $80\%$, referring to the $x$ matrix.

    \item \textbf{Chuffed:}

        The search strategy is split in two parts. In the beginning, the last column of $x$, containing the routes \textit{depot-item}, is assigned in order to address the exploration starting from the first items to deliver. Then, the rest of the path is assigned consequently.

        The best variable selection strategy found is \texttt{first\_fail} on both the steps explained above, combined with a restart policy that follows the \textit{Luby sequence}.
\end{itemize}

\subsubsection{Experimental results}

\begin{table}[ht]
    \caption{Selected subset of CP results. Results in \textbf{bold} are solved to optimality. Instances that are all solved to optimality have been omitted.}
    \label{tab:cp_results}
    \centerline{
        \begin{tabular}{c|ccccccc}
            \toprule
            & \multicolumn{4}{c}{Gecode} & \multicolumn{3}{c}{Chuffed} \\
            \cmidrule(lr){2-5} \cmidrule(lr){6-8}
            Id & \rot{base}        & \rot{implied}     & \rot{implied_lns} & \rot{SB}          & \rot{base}        & \rot{implied_rs}  & \rot{SB}          \\ 
            \midrule
            1  & \textbf{14}       & \textbf{14}       & 14                & 14                & \textbf{14}       & \textbf{14}       & \textbf{14}       \\ 
            2  & \textbf{226}      & \textbf{226}      & \textbf{226}      & \textbf{226}      & \textbf{226}      & \textbf{226}      & \textbf{226}      \\
            3  & \textbf{12}       & \textbf{12}       & 12                & 18                & \textbf{12}       & \textbf{12}       & \textbf{12}       \\
            4  & \textbf{220}      & \textbf{220}      & \textbf{220}      & \textbf{220}      & \textbf{220}      & \textbf{220}      & \textbf{220}      \\
            5  & \textbf{226}      & \textbf{226}      & \textbf{226}      & \textbf{226}      & \textbf{226}      & \textbf{226}      & \textbf{226}      \\
            6  & \textbf{226}      & \textbf{226}      & \textbf{226}      & \textbf{226}      & \textbf{226}      & \textbf{226}      & \textbf{226}      \\
            7  & \textbf{226}      & \textbf{226}      & \textbf{226}      & \textbf{226}      & \textbf{226}      & \textbf{226}      & \textbf{226}      \\
            8  & \textbf{226}      & \textbf{226}      & \textbf{226}      & \textbf{226}      & \textbf{226}      & \textbf{226}      & \textbf{226}      \\
            9  & \textbf{226}      & \textbf{226}      & \textbf{226}      & \textbf{226}      & \textbf{226}      & \textbf{226}      & \textbf{226}      \\
            10 & \textbf{226}      & \textbf{226}      & \textbf{226}      & \textbf{226}      & \textbf{226}      & \textbf{226}      & \textbf{226}      \\ 
            11 & 594            & 597           & 528           & 490           & 503           & --            & 963           & --            & 756           & 1669          & 1189          & 1081          \\ 
            12 & 449            & 428           & 375           & \textbf{346}  & 348           & --            & 833           & 1061          & 785           & 1605          & 706           & 899           \\ 
            13 & 648            & 704           & 656           & 616           & 610           & 624           & 1126          & 914           & 1126          & 1734          & 584           & 746           \\ 
            14 & 725            & 972           & 794           & 715           & 792           & --            & 1449          & --            & 1089          & --            & --            & --            \\ 
            15 & 659            & 901           & 765           & 738           & 803           & --            & 1292          & --            & --            & --            & --            & --            \\ 
            16 & 451            & 294           & \textbf{286}  & \textbf{286}  & \textbf{286}  & --            & 487           & 636           & 694           & 998           & 467           & 363           \\ 
            17 & 1324           & 1468          & 1119          & 1076          & 1155          & --            & --            & --            & --            & --            & --            & --            \\ 
            18 & 691            & 806           & 675           & 662           & 620           & --            & 1321          & --            & 1779          & --            & --            & --            \\ 
            19 & 554            & 412           & 336           & \textbf{334}  & \textbf{334}  & --            & 795           & 1079          & 765           & 1521          & 623           & 577           \\ 
            20 & 1139           & 1369          & 1104          & 1068          & 1075          & --            & --            & --            & --            & --            & --            & --            \\ 
            21 & 779            & 687           & 600           & 516           & 529           & --            & 2104          & 2140          & 993           & 2115          & 1226          & --            \\ 
            \bottomrule
        \end{tabular}
    }
\end{table}

For Gecode, we experimented different search approaches by testing restart methods, large neighborhood search (LNS), and varying symmetry breaking constraints. We observed that, compared to a depth-first approach, by simply restarting search following the Luby sequence there is a mixed impact on the results with both positive and negative effects. Instead, by also using LNS, results tend to improve globally and we observed that a higher retain probability, with a peak at around $95\%$, works better. Regarding symmetry breaking constraints, we observed that they tend to worsen the final objective value in the majority of the cases, most reasonably due to the fact that the overhead to impose them is higher than the speed-up in search. For a visual comparison, we show in \Cref{fig:cp_obj_plots} the evolution of the objective value during search. It can be seen that without restart the objective stops improving in the early stages of search as it most likely gets ``stuck" in a branch of the search tree. By introducing some non-determinism, the objective reaches lower values, with LNS being the fastest and best performing.

\begin{figure}[H]
    \centering
    \begin{subfigure}{0.49\linewidth}
        \centering
        \includegraphics[width=\linewidth]{img/cp/obj-plot_inst12.pdf}
        \caption{Instance 12}
    \end{subfigure}
    \hfill
    \begin{subfigure}{0.49\linewidth}
        \centering
        \includegraphics[width=\linewidth]{img/cp/obj-plot_inst16.pdf}
        \caption{Instance 16}
    \end{subfigure}
    \caption{Intermediate solutions using Gecode}
    \label{fig:cp_obj_plots}
\end{figure}

For Chuffed, we only experimented with restarts and symmetry breaking constraints as LNS is not available through MiniZinc. Moreover, as indicated in the documentation\footnote{\url{https://docs.minizinc.dev/en/stable/solvers.html}}, we tried enabling free search mode but only obtained worse results. Compared to Gecode, results on bigger instances are generally all worse, while smaller instances tend to be solved faster by Chuffed. As in Gecode, restarts, which we only experimented with random variable selection as it is the only non-deterministic search strategy available, and symmetry breaking constraints both yield mixed effects on the final result.

Finally, for OR-Tools, we conducted fewer experiments as it supports fewer MiniZinc annotations. In this case, we observed that enabling free search mode allows obtaining better results and, similarly to Gecode and Chuffed, symmetry breaking constraints have mixed results. As a side note, we must also observe that, as suggested in the documentation, these results are worse as OR-Tools performs better with multi-threading.
    \section{SAT model}

% The resolution of the problem is approached following two models.

% \begin{itemize}
%     \item \textbf{Unified model}:
%     based on the definition of two decision variables \texttt{assignments} and \texttt{paths}.
%     \item \textbf{Matrix model}:
%     based on the definition of one decision variable $X$.
% \end{itemize}

% Some modifications were applied to those model in order to visualize eventual differences in terms of performances.

% The discussion is going to be much more focused on the Unified model just to mantain a logical thread within the explication of models used for other solvers.

% \subsection{Unified Model}

% \paragraph*{General Definition}
% The Unified model aims to find the correct assignment to both decision variables, in such a way that it satisfy all constraints.\\
% The idea behind this kind of model is based on the separation of the two task just specified in one.

% \paragraph*{Original Workflow}
% The original workflow follows the next steps:
% \begin{enumerate}
%     \item find satisfying values for the \texttt{assignments} variable, else end the algorithm
%     \item find satisfying values for the \texttt{paths} variable, else proceed to step $4$
%     \item repeat step $2$ to find a new optimized solution,
%     \item repeat step $1$ just to find another assignment
% \end{enumerate}

% \paragraph*{New Workflow}
% The new algorithm simply tries to optimize the assignment to both variable.

% \paragraph*{Pro}
% The model was designed to improve certain intrinsic problems of the definition of a problem through SAT:
% \begin{itemize}
%     \item \textbf{Specificity of constraints}: constraining much more the assignments of the two variable guarantes to mantain lower width of exploration of the resolution tree.
% \end{itemize} 

% \paragraph*{Contro}
% The model falls into a few issues such:

% \begin{itemize}
%     \item \textbf{Dimension}: being based on two decision variables, the dimension of the problem scale exponentially with them. \footnote{The scaling dimension of the problem implies higher needs of time to build the model.}
% \end{itemize}

\subsection{Decision variables}

For SAT, we defined two different models:
\begin{description}
    \item[Unified model] based on the definition of the two decision variables $A$ (\texttt{assignments} in Z3) and $P$ (\texttt{paths} in Z3) as defined in \Cref{sec:intro}.

    \item[Matrix model] based on the definition of a single matrix $X$ such that $X\texttt{[$c$, $p$, $k$]} = 1$ iff courier $c$ delivers item $p$ as its $k$-th package.
\end{description}

Our discussion will be focused on the former to maintain coherence with the models defined in the other methods and obtain more comparable results. Its decision variables are the following:

\begin{itemize}
    \item $A$ is an $n \times m$ matrix such that $A\texttt{[$p$, $c$]} = 1$ iff courier $c$ delivers item $p$.

    \item $P$ is an $m \times (n+1) \times (n+1)$ matrix such that $P\texttt{[$c$, $loc_1$, $loc_2$]} = 1$ iff courier $c$ moves from location $loc_1$ to $loc_2$.

    \item $U$ is an $m \times n \times n$ matrix to implement MTZ subtour elimination \cite{mtz_subtour}. $U\texttt{[$c$, $p$, $k$]} = 1$ iff for the courier $c$ the item $p$ is delivered as the $k$-th.
\end{itemize}

\subsection{Objective function}

The objective function is computed as follows:
\begin{equation}
    \label{eq:obj_fun}
    \max_{c \in [1, m]}
    \sum_{loc_1=1}^{n+1} \sum_{\substack{loc_2=1,\\loc_2 \neq loc_1}}^{n+1} \texttt{D[$loc_1$, $loc_2$]} \cdot P\texttt{[$c$, $loc_1$, $loc_2$]}
\end{equation}


\subsection{Constraints}

\paragraph*{Assignment related constraints}

\begin{itemize}
    \item Capacity constraint:
    \begin{itemize}
        \item The sum of the size of the items delivered by a courier must be within its load limit:
        \begin{equation}
            \label{eq:capacity1}
            \forall c \in [1, m]:
            \sum_{p=1}^{n} A\texttt{[$p$, $c$]} \cdot s\texttt{[$p$]} \leq \texttt{$l$[$c$]}
        \end{equation}
        \item Each item must be delivered by only one courier:
        \begin{equation}
            \label{eq:capacity2}
            \forall p \in [1, n], \exists! c \in [1, m]: A\texttt{[$p$, $c$]} = 1
        \end{equation}
    \end{itemize}
\end{itemize}

\paragraph*{Path related constraints}

\begin{itemize}
    \item General path constraints
    \begin{itemize}
        \item If a courier delivers at least one package, there must exist a destination from $\texttt{DEPOT}=n+1$ with its value set to true, else it stays in \texttt{DEPOT}:
            \begin{equation}
                \label{eq:gen_path_constr1}
                \makebox[\displaywidth]{$
                    \forall c \in [1, m],
                    \forall p \in [1, n]:
                    \begin{cases}
                        P\texttt{[$c$, DEPOT, DEPOT]}=0 & \text{if } \sum_{p=1}^{n} A\texttt{[$p$, $c$]} \geq 1\\
                        P\texttt{[$c$, DEPOT, DEPOT]}=1 & \text{if } \sum_{p=1}^{n} A\texttt{[$p$, $c$]} = 0 % or otherwise
                    \end{cases}
                $}
            \end{equation}
        
        \item If a courier delivers item $p$, its successor in $P$ must be different from $p$:
        \begin{equation}
            \label{eq:gen_path_constr2}
            \forall c \in [1, m],
            \forall p \in [1, n]:
            \begin{cases}
                P\texttt{[$c$, $p$, $p$]}=0 & \text{if } A\texttt{[$p$, $c$]}=1\\
                P\texttt{[$c$, $p$, $p$]}=1 & \text{if } A\texttt{[$p$, $c$]}=0 % or otherwise
            \end{cases}
        \end{equation}

        \item For each courier, there is exactly one successor location:
        \begin{equation}
            \label{eq:gen_path_constr3}
            \forall c \in [1, m],
            \forall loc_1 \in [1, n+1]:
            \quad
            \sum_{loc_2=1}^{n+1} P\texttt{[$c$, $loc_1$, $loc_2$]} = 1
        \end{equation}

        \item For each courier, there is exactly one predecessor location:
        \begin{equation}
            \label{eq:subtour_constr1}
            \forall c \in [1, m],
            \forall loc_2 \in [1, n+1]:
            \quad
            \sum_{loc_1=1}^{n+1} P\texttt{[$c$, $loc_1$, $loc_2$]} = 1
        \end{equation}
    \end{itemize}

    \item MTZ subtour elimination constraints
        \begin{itemize}
            \item $U$ relative to the first item $p$ of the path of a courier $c$ must be 1:
            \begin{equation}
                \label{eq:subtour_constr2}
                \forall c \in [1, m],
                \forall p \in [1, n]:
                P\texttt{[$c$, DEPOT, $p$]}=1
                \Rightarrow
                U\texttt{[$c$, $p$, $1$]}=1
            \end{equation}
            
            \item $U$ relative to an item $j$ delivered after $i$ should be greater ($U_j \geq U_i+1$):
            \begin{equation}
                \makebox[\displaywidth]{$
                    \label{eq:subtour_constr3}
                    \forall c \in [1, m], \ \forall i, j, k \in [1, n]:
                    \left(P\texttt{[$c$, $i$, $j$]} \land U\texttt{[$c$, $i$, $k$]}\right) \Rightarrow
                    \sum_{l=k+1}^{n} U\texttt{[$c$, $j$, $l$]} = 1
                $}
            \end{equation}
            
            \item $U$ relative to an item $p$ of a courier must have exactly one value associated:
            \begin{equation}
                \label{eq:subtour_constr4}
                \forall c \in [1, m],
                \forall p \in [1, n]:
                \sum_{k=1}^{n} U\texttt{[$c$, $p$, $k$]} = 1
            \end{equation}

            \item MTZ formulation:
            \begin{equation}
                \label{eq:subtour_constr5}
                \makebox[\displaywidth]{$
                    \begin{split}
                        &\forall c \in [1, m], \ \forall i, j, k_1, k_2 \in [1, n]: \\
                        &(U\texttt{[$c$, $i$, $k_1$]} \land U\texttt{[$c$, $j$, $k_2$]})\Rightarrow
                        (k_1 - k_2 + 1) \leq (n-1) \cdot (1 - P\texttt{[$c$, $i$, $j$]})
                    \end{split}
                $}
            \end{equation}
        \end{itemize}
\end{itemize}

\subsubsection{Symmetry breaking constraint}

A possible way to reduce tree exploration is to introduce symmetry breaking constraints. In particular, for SAT, we experimented only with the one defined in \Cref{eq:cp_symm_packs}.


\subsection{Validation}

\subsubsection{Experimental design}

Some modifications were applied to the basic model in order to visualize possible differences in performance.
The original Unified Model was modified in three different versions:
\begin{enumerate*}[label=(\roman*)]
    \item with Symmetry Breaking Constraints,
    \item with Cumulative Constraint Application,
    \item with the Heule Encoding for \texttt{at\_most\_one()}.
\end{enumerate*}

To optimize the objective function, we follow the standard approach of iteratively solving the problem and imposing as new constraint the fact that a new objective must be lower than the one already found.


\subsubsection{Experimental results}

As we can notice from \Cref{tab:sat_results}, the performances were not very much different from the basic model. Only for the first 10 instances it has been possible to reach at least a suboptimal solution, while for the remaining ones the construction of the model required too much time causing to exceed the timeout limit. Symmetry breaking also did not contribute to improve the results. 

\begin{table}[H]
    \centering
    \caption{SAT results. Results in \textbf{bold} are solved to optimality. Instances that are all without a solution have been omitted.}
    \label{tab:sat_results}
    \centerline{
        \begin{tabular}{cccccc}
            \toprule
            Id & un-model & un-symm-model & un-cum-constr-model & un-heule-enc-model & matrix-model \\ 
            \midrule
            1 & \textbf{14} &       \textbf{14} &   \textbf{14} &   \textbf{14} &   \textbf{14} \\ 
            2 & \textbf{226} &      \textbf{226} &  \textbf{226} &  \textbf{226} &  \textbf{226} \\ 
            3 & \textbf{12} &       \textbf{12} &   \textbf{12} &   \textbf{12} &   \textbf{12} \\ 
            4 & \textbf{220} &      \textbf{220} &  \textbf{220} &  \textbf{220} &  \textbf{220} \\ 
            5 & \textbf{206} &      \textbf{206} &  \textbf{206} &  \textbf{206} &  \textbf{206} \\ 
            6 & \textbf{322} &      \textbf{322} &  \textbf{322} &  \textbf{322} &  \textbf{322} \\ 
            7 & 232 &       238 &   222 &   296 &   292 \\ 
            8 & \textbf{186} &      \textbf{186} &  \textbf{186} &  \textbf{186} &  \textbf{186} \\ 
            9 & \textbf{436} &      \textbf{436} &  \textbf{436} &  \textbf{436} &  \textbf{436} \\ 
            10 & \textbf{244} &     \textbf{244} &  \textbf{244} &  \textbf{244} &  \textbf{244} \\ 
            \bottomrule
        \end{tabular}
    }
\end{table}


\begin{figure}[h]
    \centering
    \begin{subfigure}{0.49\linewidth}
        \centering
        \includegraphics[width=\linewidth]{img/sat/time.pdf}
        \caption{Resolution time}
    \end{subfigure}
    \hfill
    \centering
    \begin{subfigure}{0.49\linewidth}
        \centering
        \includegraphics[width=\linewidth]{img/sat/conflicts.pdf}
        \caption{Number of conflicts}
    \end{subfigure}
    \\
    \centering
    \begin{subfigure}{0.49\linewidth}
        \centering
        \includegraphics[width=\linewidth]{img/sat/max_memory.pdf}
        \caption{Maximum amount of memory required}
    \end{subfigure}
    % \hfill
    % \centering
    % \begin{subfigure}{0.49\linewidth}
        % \centering
        % \includegraphics[width=\linewidth]{img/sat/mk_bool_var.pdf}
        % \caption{Number of boolean variables}
    % \end{subfigure}
    % \hfill
    % \centering
    % \hfill
    % \centering
    % \begin{subfigure}{0.49\linewidth}
    %     \centering
    %     \includegraphics[width=\linewidth]{img/sat/restart.pdf}
    %     \caption{Number of restarts}
    % \end{subfigure}
    \caption{Statistics about the resolution of the problem for the first 10 instances}
    \label{fig:sat_plots}
\end{figure}

For all models, some statistics were extracted and are presented in \Cref{fig:sat_plots}:
\begin{enumerate*}[label=(\roman*)]
    \item \textit{time}: there are similar intervals of time for all unified model variants, while the matrix model ensures in some instances a better performance;
    \item \textit{number of conflicts}: it useful for estimating the size of the search space (the lower the better)\footnote{\url{https://stackoverflow.com/questions/17856574/how-to-interpret-statistics-z3}}. We can observe that the matrix model manages to have fewer conflicts, which is in line with its resolution time;
    \item \textit{max memory}: also in line with the previous observations, the matrix model tend to require less memory to solve the problem.
\end{enumerate*}

    \section{SMT model}


\subsection{Decision variables}

The SMT model uses the logic of quantifier-free linear integer arithmetic (\texttt{QF\_LIA}) and relies on the following decision variables:

\begin{itemize}
    \item For each package $j$, $A_j \in [1, m]$ (\texttt{ASSIGNMENTS[j]} in the code) indicates which courier delivers package $j$ where $A_j = i$ indicates that courier $i$ delivers package $j$.
    
    \item For each courier $i$, $P_{i,d} \in [1, n+1]$ for $d \in [1, n+1]$ (\texttt{PATH[i][d]} in the code) follows the definition of \Cref{eq:path_def}.

\end{itemize}

% \subsection{Auxiliary variables}
% More variables have been defined in order to make it easier to define some constraints and the objective function:

% \begin{itemize}
%     \item For each courier $i$, $D_i \in \mathbb{N}$ (\texttt{DISTANCES[i]} in Z3) is equal to the distance traveled by each courier.

%     \item For each courier $i$, $C_i \in \mathbb{N}$ (\texttt{COUNT[i]} in Z3) models the number of packages delivered by courier $i$.

%     \item For each item $j$ and for each courier $i$, $PPC_{i,j}$ (\texttt{PACKS\_PER\_COURIER[i][j]} in Z3) model the packages delivered by each courier and it is used for symmetry breaking only.
% \end{itemize}

\subsection{Objective function}
The objective function is defined as follows:

\[ \max_{i \in [1, m]} \texttt{DISTANCES}_i  \]
where $\texttt{DISTANCES}_i$ is equal to the distance traveled by each courier.

\subsection{Constraints}

\begin{itemize}
    \item Weight constraint: 
    \begin{equation}
        \forall i \in [1, m]: \sum_{j \in [1, n]: A_j = i} s_j \leq l_i 
    \end{equation}

    \item Assignment and path constraint:
    \begin{equation}
        \forall i \in [1, m], \forall j \in [1, n]: \quad A_j = i \Longleftrightarrow P_{i,j} \neq j
    \end{equation}
    \begin{equation}
        \forall i \in [1, m], \forall j \in [1, n]: \quad A_j \neq i \Longleftrightarrow P_{i,j} = j
    \end{equation}

    \item All the elements of each row of $P$ should be distinct:
    \begin{equation}
        \forall i \in [1, m]: \quad P_{i,j_1} \neq P_{i,j_2} \quad \forall j_1,j_2 \in [1, n+1], j_1 \neq j_2 
    \end{equation}

    % \item Count constraint: 

    % \begin{equation}
    %     \forall i \in \{ 1, \dots, m \}: \quad C_i = | \{ j \in \{1, \dots, n\} | A_j = i \}|
    % \end{equation}

    \item Subcircuit constraint: each row $P_i$ should define a subcircuit, that is a Hamiltonian path that ignores all the elements $P_{i,j} = j$, namely the packages that the courier $i$ do not deliver. This can be modelled through MTZ subtour elimination as defined in \Cref{eq:subtour_constr5}.

    % \item Distance constraint:
    % \begin{equation}
    %     \forall i \in \{1, \dots, m\}: \quad D_i = \sum_{j \in \{1, \dots, n+1\}} 
    %         \begin{cases}
    %             D_{j,P_{i,j}} & \text{if } P_{i,j} \neq j  \\
    %             0 & \text{if } P_{i,j} = j
    %         \end{cases}
    % \end{equation}
\end{itemize}


\subsubsection{Implied constraints}
For SMT, we experimented the implied constraint defined in \Cref{eq:impl_constr}.


\subsubsection{Symmetry breaking constraints}
For SMT, we experiment both symmetry breaking approaches as presented in \Cref{eq:cp_symm_amount,eq:cp_symm_packs}.



\subsection{Validation}


\subsubsection{Experimental design}

The experimental setup consists of two steps: first, we developed a Python package to automate the generation of SMT-LIB code from a high-level interface, allowing us to easily experiment and compare different solvers. More specifically, as solvers we experimented with Z3, cvc5, OpenSMT, SMTInterpol, and Yices 2 using both linear and binary optimization approaches. Then, to improve the performances on larger instances, we experimented with two different search strategies using the Z3 Python library (\texttt{z3py}):
\begin{description}
    \item[Two solvers approach]
        As SMT solvers do not allow to assign a priority to the variables, this approach attempts to guide the exploration of the search space by alternating two solvers: the first one finds $A$ and the second one finds $P$ given $A$. In other words, the former decides which courier delivers which package and the latter decides the route taken by each courier, basically solving $m$ different travelling salesman problems. 

    \item[Local search approach]
        Similarly to the previous one, this approach also uses two solvers to first find $A$ and then $P$. However, instead of letting the second solver find an optimal solution for $P$ on its own, it is manually guided by performing a local search starting from a trivial solution (i.e., a path that delivers the items ordered by index).
\end{description}


\subsubsection{Experimental results}

The results of our experiments are presented in \Cref{tab:smt_results}. As linear optimization always outperformed binary search, we only present the former.
Analyzing SMT-LIB results, we can observe that all five solvers perform more or less similarly. The best performing is Yices 2 which solves \texttt{QF\_LIA} logic based on the simplex algorithm \cite{yices2}. On the other hand, the worst performing is SMTInterpol which relies on Craig interpolation \cite{smtinterpol}.
Furthermore, experiments in \texttt{z3py} show that symmetry breaking and implied constraints do not provide significant contribution in improving the results.

By analyzing the two search approaches, we can observe that using two separate solvers have a negligible impact on the final results with mixed effects when using symmetry breaking constraints. Instead, local search enables the model to find a solution in a reasonable amount of time, even for the largest instances.

\begin{table}[H]
    \caption{SMT results. Results in \textbf{bold} are solved to optimality. Instances that are all solved to optimality have been omitted.}
    \label{tab:smt_results}
    \centerline{
        \begin{tabular}{c|cccccccccccc}
            \toprule
            & \multicolumn{5}{c}{SMT-LIB (plain)} & \multicolumn{7}{c}{\texttt{z3py}} \\
            \\[-3ex] % Remove vertical line gap
            \cmidrule(lr){2-6} \cmidrule(lr){7-13}
            Id          & \rot{Z3}      & \rot{cvc5}    & \rot{OpenSMT} & \rot{SMTInterpol} & \rot{Yices 2} & \rot{plain}   & \rot{\makecell[l]{plain\\[-0.3em]\small+ SB}}  & \rot{\makecell[l]{plain\\[-0.3em]\small+ IC}}  & \rot{2 solvers}   & \rot{\makecell[l]{2 solvers\\[-0.3em]\small+ SB}}  & \rot{\makecell[l]{2 solvers\\[-0.3em]\small+ IC}}  & \rot{Local search}   \\ 
            \midrule                                     
            7           & 228           & 210           & 218           & 372               & \textbf{167}  & 174           & 168               & 181               & \textbf{167}      & \textbf{167}          & \textbf{167}          & \textbf{167}          \\ 
            9           & \textbf{436}  & \textbf{436}  & \textbf{436}  & 437               & \textbf{436}  & \textbf{436}  & \textbf{436}      & \textbf{436}      & \textbf{436}      & \textbf{436}          & \textbf{436}          & \textbf{436}          \\ 
            10          & \textbf{244}  & \textbf{244}  & \textbf{244}  & 381               & \textbf{244}  & \textbf{244}  & \textbf{244}      & \textbf{244}      & \textbf{244}      & \textbf{244}          & \textbf{244}          & \textbf{244}          \\ 
            11          & --            & --            & --            & --                & --            & --            & --                & --                & --                & --                    & --                    & 547                   \\ 
            12          & --            & --            & --            & --                & --            & --            & --                & --                & --                & --                    & --                    & 435                   \\ 
            13          & 1446          & --            & --            & --                & 1490          & --            & --                & --                & 1812              & 1346                  & 1832                  & 632                   \\ 
            14          & --            & --            & --            & --                & --            & --            & --                & --                & --                & --                    & --                    & 1177                  \\ 
            15          & --            & --            & --            & --                & --            & --            & --                & --                & --                & --                    & --                    & 1140                  \\ 
            16          & --            & --            & --            & --                & 1032          & --            & --                & --                & 1510              & 1944                  & 1861                  & 303                   \\ 
            17          & --            & --            & --            & --                & --            & --            & --                & --                & --                & --                    & --                    & 1525                  \\ 
            18          & --            & --            & --            & --                & --            & --            & --                & --                & --                & --                    & --                    & 917                   \\ 
            19          & --            & --            & --            & --                & --            & --            & --                & --                & --                & 870                   & --                    & 398                   \\ 
            20          & --            & --            & --            & --                & --            & --            & --                & --                & --                & --                    & --                    & 1378                  \\ 
            21          & --            & --            & --            & --                & --            & --            & --                & --                & --                & --                    & --                    & 648                   \\ 
            \bottomrule
        \end{tabular}
    }
\end{table}
    \section{MIP model}


\subsection{Decision variables}

The MIP models also follow the same idea presented in \Cref{sec:intro}.
% In particular, taking inspiration from the AMPL book \cite{AMPLbook}, we model the Hamiltonian cycle by using a binary tensor that encodes the Hamiltonian cycle of each courier through all the possible delivery points, starting and ending at the depot. 
% This travel is an Hamiltonian cycle and we recall here the definition:
% \begin{definition}[Hamiltonian cycle]
% In the mathematical field of graph theory an Hamiltonian cycle (or Hamiltonian circuit) is a cycle that visits each vertex exactly once.
% \end{definition}
% In order to preserve the properties of the Hamiltonian cycle and to avoid the possible presence of subcircuits we added some constraints, and in particular for the last case we followed the Miller–Tucker–Zemlin (MTZ) approach.\\
The models are based on the following three variables:
\begin{itemize}
    \item A binary tensor $X \in \{0,1\}^{(n+1) \times (n+1) \times m}$, where $X[i,j,k] = 1$ if and only if the courier $k$ departs from the $i$-th delivery point and arrives at the $j$-th one. This formulation is inspired from the AMPL book \cite{AMPLbook}.

    \item A binary matrix $A \in \{0,1\}^{n \times m}$, where $A[i,k] = 1$ if and only if the package $i$ is delivered by the courier $k$. We observe that these variables are not strictly necessary for the model, but they allowed us to write some constraints in an easier way.

    \item An auxiliary matrix $u \in \{1,\dots,n+1\}^{(n+1) \times m}$ that keeps track of the order in which the nodes are visited by each courier starting from $1$. It is necessary, following the MTZ formulation, for subcircuit elimination. The interpretation is that, fixed a courier $k$, $u[i,k] < u[j,k]$ implies that the node $i$ is visited before the node $j$ by the courier $k$.
\end{itemize}


\subsection{Objective function}
As objective function, defined from the assignments as the maximum distance travelled by any courier, we use the following: 
\begin{equation}
    \max_{k \in 1 \dots m}\sum_{i = 1}^{n+1} X[i,j,k] \cdot D[i,j]
\end{equation}
% As upper bound we chose the sum over all the indexes of the matrix $D$: $\sum_{i,j = 1}^{n+1} D[i,j]$. Instead as lower bound we chose the maximum distance travelled picking only one package: $\max_{i \in 1 \dots n} (D[n+1,i] + D[i,n+1])$. In fact, due to the triangular inequality, if another package is picked up by the courier the distance will grow. 

\subsection{Constraints}

We defined the following constraints:
\begin{itemize}
    \item Constraint to link the variables $A$ and $X$:
    \begin{equation}
        \sum_{j = 1}^{n+1} X[i,j,k] = A[i,k]  \qquad  \forall i \in 1 \dots n,\forall k \in 1 \dots m.
    \end{equation}
    \begin{equation}
        \sum_{i = 1}^{n+1} X[i,j,k] = A[j,k]  \qquad  \forall j \in 1 \dots n,\forall k \in 1 \dots m.
    \end{equation}

    \item Constraint to guarantee that each package is assigned:
    \begin{equation}
        \sum_{k = 1}^{m} A[i,k] = 1  \qquad \forall i \in 1 \dots n.
    \end{equation}

    \item Constraint for the capacity of each courier:
    \begin{equation}
        \sum_{i = 1}^{n} A[i,k] s[i] \leq l[k] \qquad \forall k \in 1 \dots m.
    \end{equation}

    \item Constraint to avoid that each courier departs and arrives at the same point:
    \begin{equation}
        X[i,i,k] = 0 \qquad \forall i \in 1 \dots n+1, \forall k \in 1 \dots m.
    \end{equation}

    % \item Two constraints to ensure that there is one arrival and one departure for each node, respectively. This concerns only the internal nodes because the depot is visited by each courier:
    % \begin{equation}
    %     \sum_{i \in 1 \dots n+1, k \in 1 \dots m} X[i,j,k] = 1 \qquad \forall j \in 1 \dots n.  
    % \end{equation}
    % \begin{equation}
    %     \sum_{j \in 1 \dots n+1, k \in 1 \dots m} X[i,j,k] = 1 \qquad \forall i \in 1 \dots n.  
    % \end{equation}

    \item Constraint for the preservation of the flow (if one courier arrives at one node, he departs from the same one):
    \begin{equation}
        \sum_{i = 1}^{n+1} X[i,j,k] = \sum_{i = 1}^{n+1} X[j,i,k] \qquad \forall j \in 1 \dots n, \forall k \in 1 \dots m.
    \end{equation}

    \item Constraints to ensure that each courier starts and ends its route at the depot:
    \begin{equation}
        \sum_{j = 1}^{n} X[n+1,j,k] = 1 \qquad \forall k \in 1 \dots m.
    \end{equation}
    \begin{equation}
        \sum_{j = 1}^{n} X[j,n+1,k] = 1 \qquad \forall k \in 1 \dots m.
    \end{equation}

    \item Constraints for MTZ subtour elimination:
    \begin{itemize}
        \item MTZ condition:
            \begin{equation}
            \makebox[\displaywidth]{$
                u[i,k] - u[j,k] + 1 \leq (n-1)(1 - X[i,j,k]) \qquad \forall i \in 1 \dots n, \forall j \in 1 \dots n+1, \forall k \in 1 \dots m.
            $}
            \end{equation}

        \item $u[i, k] = 1$ iff $i$ is the first item delivered by $k$:
            \begin{equation}
            \makebox[\displaywidth]{$
                u[i,k] \leq X[n+1,i,k] + (n+1)(1-X[n+1,i,k]) \qquad \forall i \in 1 \dots n, \forall k \in 1 \dots m.
            $}
            \end{equation}

        \item $u[j, k] \geq u[i, k] + 1$ iff $i$ is delivered before $j$:
            \begin{equation}
            \makebox[\displaywidth]{$
                u[j,k] \geq (u[i,k] + 1)X[i,j,k] \qquad \forall i \in 1 \dots n, \forall j \in 1 \dots n+1, \forall k \in 1 \dots m.
            $}
            \end{equation}
    \end{itemize}
    
    \subsubsection{Implied Model}
    For the implied model, we added one more constraint with the same meaning of \Cref{eq:impl_constr}:
    \begin{equation}
        \sum_{i = 1}^{n} A[i,k] \geq 1 \qquad \forall k \in 1 \dots m.
    \end{equation}
    
    \subsubsection{Symmetry Model}
    For the symmetry model, we added the symmetry breaking constraint related to the number of delivered packages as defined in \Cref{eq:cp_symm_amount}:
    \begin{equation}
        \sum_{i = 1}^{n} A[i,k] \leq \sum_{i = 1}^n A[i,j],
    \end{equation}
    where $k,j \in 1, \dots, m$ with $k < j$ and $l[k] = l[j]$.
\end{itemize}


\subsection{Validation}

\subsubsection{Experimental design}

For the MIP models, we choose to use the solver-independent language AMPL. The workflow is based on the construction of three different models: the initial one, the implied one, and the symmetry breaking one.

For reproducibility, we decided to only use open-source solvers provided by the AMPL framework such as HiGHS, SCIP, and GCG.


\subsubsection{Experimental results}

Starting from some preliminary experiments, we immediately observed that GCG performs poorly on the first ten instances and therefore decided to discard it from the full experiments.

In \Cref{tab:mip_results}, we present the results of the MIP models. We can observe that the SCIP solver with the addition of implied and symmetry breaking constraints improves in performance. On the other hand, this behavior is not the same for HiGHS.
Nevertheless, the HiGHS solver performances are in general significantly better than the SCIP ones.

\begin{table}[H]
    \centering
    \caption{MIP results. Results in \textbf{bold} are solved to optimality. Instances that are all without a solution have been omitted.}
    \label{tab:mip_results}
    \centerline{
        \begin{tabular}{ccccccc}
            \toprule
            Id & initial-scip & initial-highs & implied-scip & implied-highs & symmetry-scip & symmetry-highs \\ 
            \midrule
            1 & \textbf{14} &       \textbf{14} &   \textbf{14} &   \textbf{14} &   \textbf{14} &   \textbf{14} \\ 
            2 & \textbf{226} &      \textbf{226} &  \textbf{226} &  \textbf{226} &  \textbf{226} &  \textbf{226} \\

            3 & \textbf{12} &       \textbf{12} &   \textbf{12} &   \textbf{12} &   \textbf{12} &   \textbf{12} \\ 
            4 & \textbf{220} &      \textbf{220} &  \textbf{220} &  \textbf{220} &  \textbf{220} &  \textbf{220} \\

            5 & \textbf{206} &      \textbf{206} &  \textbf{206} &  \textbf{206} &  \textbf{206} &  \textbf{206} \\

            6 & \textbf{322} &      \textbf{322} &  \textbf{322} &  \textbf{322} &  \textbf{322} &  \textbf{322} \\

            7 & \textbf{167} &      \textbf{167} &  \textbf{167} &  \textbf{167} &  \textbf{167} &  \textbf{167} \\

            8 & \textbf{186} &      \textbf{186} &  \textbf{186} &  \textbf{186} &  \textbf{186} &  \textbf{186} \\

            9 & \textbf{436} &      \textbf{436} &  \textbf{436} &  \textbf{436} &  \textbf{436} &  \textbf{436} \\

            10 & \textbf{244} &     \textbf{244} &  \textbf{244} &  \textbf{244} &  \textbf{244} &  \textbf{244} \\

            13 & 642 &      726 &   616 &   694 &   526 &   692 \\
            16 & -- &       \textbf{286} &  -- &    557 &   -- &    320 \\
            \bottomrule
        \end{tabular}
    }
\end{table}


For the first ten instances, we plotted the graphs about the execution time (in seconds), the number of simplex iterations, and branching nodes explored by these two solvers.
\begin{figure}[ht]
    \centering
    \begin{subfigure}{0.8\linewidth}
        \centering
        \includegraphics[width=\linewidth]{img/mip/time.pdf}
        % \includegraphics[width=\textwidth]{img/mip/simplex_iterations_2.png}
    \end{subfigure}
    \begin{subfigure}{0.8\linewidth}
        \centering
        % \includegraphics[width=\textwidth]{img/mip/Figure_2.png}
        \includegraphics[width=\linewidth]{img/mip/simplex.pdf}
\end{subfigure}
    \caption{Compared statistics of the performances of the three models}
\end{figure}

% \begin{figure}[H]
%     \centering
%     \includegraphics[width=\linewidth]{img/mip/simplex.pdf}
%     \caption{Compared statistics of the performances of the three models, divided by SCIP and HiGHS solvers.}
% \end{figure}


On a theoretical level, SCIP and HiGHS try to initially solve the relaxed-version of the problem (LP) using the revised simplex-method and finding a lower bound for the solution; then, if the solution found is not an integer, they start to solve the MIP part using branch-and-cut (SCIP) or branch-and-bound (HiGHS). For the resolution of the sub-problems generated by these two methods, they proceed recursively by applying the same algorithm. Therefore, from the plots, we can observe that a lower number of simplex iterations and branching nodes corresponds to a faster resolution time. This is in line with our previous observation regarding HiGHS as a better performing solver.



    \section{Conclusions}

    We experimented several models using different methods and attempted to implement the same core idea across the whole project to make results comparable. Overall, all approaches are able to solve the smaller instances while, for bigger instances, only CP and SMT were able to at least produce a suboptimal solution by using proper search heuristics. Moreover, for this problem, we surprisingly noted that symmetry breaking constraints generally tend to, except in a few cases, worsen the results. To conclude, we can observe that, for this formulation of the problem, being able to guide the solver when exploring the search space is one of the most important factors to obtain good results.

    \printbibliography

    \begin{appendices}
        \section{Implied constraint proof} \label{sec:impl_proof}
        \begin{lem}
            Assuming that the capacity of each courier allows delivering at least a package, if there exists an optimal solution, then there exists an optimal solution where each courier delivers at least one package.
        \end{lem}
        \begin{proof}
            Let us assume that the optimal solution is $D_j$ and there is a courier, say $k_1$, which do not deliver any package. Let us also suppose that the courier $k_j$ is the one that covers the maximum distance $D_j$. If we assign one package that $k_j$ brings, say $i$, to $k_1$, then, due to the triangle inequality, the two new distances $D_1$, travelled by the courier $k_1$ with $i$, and $D_2$, travelled by the courier $k_j$ without $i$, are less or equal to $D_j$. In fact:
            \begin{equation}
                \begin{split}
                    D_1 = &D[\texttt{depot},i] + D[i,\texttt{depot}] \leq\\
                        & D[\texttt{depot}, i_1] + \dots + D[i_r, i] + D[i, i_s] + \dots + D[i_t, \texttt{depot}] = D_j.
                \end{split}
            \end{equation}
            \begin{equation}
                \begin{split}
                    D_2 = &D[\texttt{depot},i_1] + \dots + D[i_r,i_s] + \dots + D[i_t, \texttt{depot}] \leq\\
                    &D[\texttt{depot}, i_1] + \dots + D[i_r, i] + D[i, i_s] + \dots + D[i_t, \texttt{depot}] = D_j.
                \end{split}
            \end{equation}
            Therefore, there are the following cases:
            \begin{itemize}
                \item If $D_1 = D_2$, either both $k_1$ and $k_j$ cover the maximum distance or neither of them do, and another courier has a route of cost $D_j$.
                \item If $D_1 > D_2$, either $k_1$ covers the maximum distance or another courier that is not $k_1$ and $k_j$ does.
                \item If $D_1 < D_2$, same as above for $k_j$.
            \end{itemize}
            So, an optimal solution still exists and has as objective value $D_j$.
        \end{proof}
    \end{appendices}

\end{document}